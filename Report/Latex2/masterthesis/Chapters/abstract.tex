{\chapter*{Abstract}}
Bubbles have fascinated and baffled many canny observers of financial markets. The  extreme fluctuations in Bitcoin (BTC/USD) price, hence bubbles, has created interest among media, regulators and researchers to understand underlying cause. The extreme price volatility demands a thorough investigation of transactions network in bitcoin. With readily available blockchain, log of all transactions that were ever verified on the Bitcoin network, we attempt to investigate how structural changes in the network accompany significant changes in the exchange price of bitcoins. The results of this investigation is baseline for understanding bitcoin price formation or forecasting BTC/USD exchange price.

In big data setting, we are trying to understand "How much is a large-scale graph (bitcoin transaction graph) transformed over time or by a significant event (bubbles)?" or "How structurally similar are two large-scale graph (bitcoin transactions graphs)?". In application coming out from dynamic graph, similarity in graph connectivity is great tool to detect
anomalies in the bitcoin network that are manifest in the form of bubbles.
 
Taking help of graph kernels, a polynomial alternative of graph isomorphism problem, we find an algorithm to calculate the similarity index ($0-1$) between the two graphs. This is quantitative measure of transformation over time. By extending the quite novel framework for graph kernels inspired by latest advancements in natural language processing and deep learning, we propose, deep graph propagation kernels. The unseen deep framework in the literature takes account of attributed graphs with continuous values.The deep graph kernels outperforms its best base graph kernels variants in terms of capturing correlation between newtork structure and market price. This defines a baseline to predict the price of BTC/USD exchange, which leverage on deep learning learning framework to extract feature space of blockchain.
 


